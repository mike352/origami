\documentclass{article}
\usepackage{graphicx}
\usepackage[margin=0cm,a4paper]{geometry} %global setting. 
\usepackage{tikz}
\usetikzlibrary{calc}

\begin{document}
%The TikZ environment is in cm
%A0: 84.1 x 118.9
%A1: 59.4 x 84.1
%A2: 42.0 x 59.4
%A3: 29.7 x 42.0
%A4: 21.0 x 29.7
\def\AW{21.0} %length in cm of page width, should match geomtry paper size
\def\AL{29.7} %length in cm of page length, should match geomtry paper size
\def\gridmargin{1} %page margin size
\def\printoffset{0.1} %offset from paper margin in cm
\def\unitsize{4} %spacing between parallel lines in cm

\pgfmathsetmacro{\doubleunit}{2*\unitsize}
\pgfmathsetmacro{\fourunit}{4*\unitsize}
\pgfmathsetmacro{\ALL}{2*\AL}
\begin{figure*}[htpb]
\begin{center}
\begin{tikzpicture}
\clip (\gridmargin-\printoffset,\gridmargin-\printoffset) rectangle (\AW-\gridmargin+\printoffset,\AL-\gridmargin+\printoffset);
\begin{scope}[shift={(\gridmargin,\gridmargin)}]
\foreach \y in {0,\unitsize,...,\AL}
	{\draw[line width = 1pt] (0,\y) -- (\AW,\y);
	\draw[line width = 1pt] (\y,0) -- (\y,\AL);}
\foreach \x in {0,\doubleunit,...,\ALL}
	{\draw[line width = 1pt] (\x,0) -- ({-\AL+\x},\AL);}
\foreach \x in {0,\doubleunit,...,\AW}
	{\draw[line width = 1pt] (\x,0) -- ({\x+\AL},\AL);}
\foreach \x in {-\doubleunit,-\fourunit,...,-\ALL}
	{\draw[line width = 1pt] (\x,0) -- ({\x+\AL},\AL);}
\end{scope}
\end{tikzpicture}
\end{center}
\end{figure*}

\end{document}