\documentclass{article}
\usepackage{graphicx}
\usepackage[margin=0cm,a0paper]{geometry} %global setting 
\usepackage{tikz}
\usetikzlibrary{calc}

\begin{document}
%The TikZ environment is in cm
%The geometry package sets the margins
%A0: 84.1 x 118.9
%A1: 59.4 x 84.1
%A2: 42.0 x 59.4
%A3: 29.7 x 42.0
%A4: 21.0 x 29.7
\def\AW{84.1} %length in cm of page width, should match geomtry paper size
\def\AL{118.9} %length in cm of page length, should match geomtry paper size
\def\gridmargin{1} %page margin size
\def\printoffset{0.1} %offset from paper margin in cm
\def\gridnum{32} %number of rays
\def\numpleats{9} %total number of pleats
\def\spacing{7} %spacing of alternate rays in cm, outer ones
\def\offset{4.5} %initial offset of alternate inner rays in cm

%Total length of ray is paper width minus twice the paper margin
\pgfmathsetmacro{\raysize}{\AW-2*\gridmargin}
%The size of the inner pleat spacings is calculated based on spacing and offset
\pgfmathsetmacro{\innerspacing}{\spacing*(\raysize-\offset)/\raysize} 
\begin{figure*}[htpb]
\begin{center}
\begin{tikzpicture}
\clip (\gridmargin-\printoffset,\gridmargin-\printoffset) rectangle (\AW-\gridmargin+\printoffset,\AL-\gridmargin+\printoffset);
\begin{scope}[shift={(\gridmargin,\gridmargin)}]
\draw (0:\raysize) arc (0:90:\raysize);
\foreach \y in {0,1,...,\gridnum}
	{\draw[line width = 1pt] (0,0) -- ({90*\y/\gridnum}:\raysize);}
\foreach \x in {1,2,...,\numpleats}
	\foreach \y in {2,4,...,\gridnum}
	{\draw[line width = 1pt] ({90/\gridnum*(\y-2)}:{\raysize-(\x-1)*\spacing}) -- ({90/\gridnum*(\y-1)}:{\raysize-\offset-(\x-1)*\innerspacing});
	\draw[line width = 1pt] ({90/\gridnum*\y}:{\raysize-(\x-1)*\spacing}) -- ({90/\gridnum*(\y-1)}:{\raysize-\offset-(\x-1)*\innerspacing});}
\end{scope}
\end{tikzpicture}
\end{center}
\end{figure*}

\end{document}