\documentclass{article}
\usepackage{graphicx}
\usepackage[margin=0cm,a3paper]{geometry} %MATCH PAPER SIZE BELOW, landscape optional
\usepackage{tikz}
\usetikzlibrary{calc}

\begin{document}
%The TikZ environment is in cm
%The geometry package sets the margins
%Sizes for \AW and \AL (swap for landscape):
%a0paper: 84.1 x 118.9
%a1paper: 59.4 x 84.1
%a2paper: 42.0 x 59.4
%a3paper: 29.7 x 42.0
%a4paper: 21.0 x 29.7
%letterpaper:  21.6 x 27.9
\def\AW{29.7} %length in cm of page width, **MUST MATCH GEOMETRY PAPER SIZE ABOVE**
\def\AL{42.0} %length in cm of page length, **MUST MATCH GEOMETRY PAPER SIZE ABOVE**
\def\gridmargin{1} %page margin size
\def\printoffset{0.1} %offset from paper margin in cm

\def\qparam{3} %parameter determining the angles: q>1 (required), maybe >2.735 (prevvents overlap of pleats)
\def\rparam{2} %parameter determining the ratio of the rectangular gaps, it will be 1:\rparam
\def\vpleatwidth{2} %width of the final vertical pleats 
\def\hpleatwidth{2} %width of the final horizontal pleats 
\def\scaleparam{1} %scaling

\pgfmathsetmacro{\hwidth}{(\qparam+\rparam)/(\qparam*\qparam-1)} %width of horizontal trapezoids
\pgfmathsetmacro{\vwidth}{(\qparam*\rparam+1)/(\qparam*\qparam-1)} %width of vertical trapezoids
\pgfmathsetmacro{\yunitsize}{2*(\rparam+\hpleatwidth+2*\hwidth)}
\pgfmathsetmacro{\xunitsize}{2*(1+\vpleatwidth+2*\vwidth)}
\pgfmathsetmacro{\YMAX}{ceil(\AL/\yunitsize)}
\pgfmathsetmacro{\XMAX}{ceil(\AW/\xunitsize)}

\begin{figure*}[htpb]
\begin{center}
\begin{tikzpicture}
\clip (\gridmargin-\printoffset,\gridmargin-\printoffset) rectangle (\AW-\gridmargin+\printoffset,\AL-\gridmargin+\printoffset);
\begin{scope}[shift={(\gridmargin,\gridmargin)}]
\foreach \y in {0,1,...,\YMAX}{
	\foreach \x in {0,1,...,\XMAX}{
		\begin{scope}[scale=\scaleparam,shift={(\x*\xunitsize,\y*\yunitsize)}]{
		\draw[line width = 1pt] (\vwidth,0) -- (0,\rparam+\hwidth) -- (1+\vwidth,\rparam+2*\hwidth) -- (1+2*\vwidth,\hwidth) -- (\vwidth,0);
		\draw[line width = 1pt] (\vwidth,0) -- (1+\vpleatwidth+2*\vwidth,0);
		\draw[line width = 1pt] (1+2*\vwidth,\hwidth) -- (1+\vpleatwidth+2*\vwidth,\hwidth);
		\draw[line width = 1pt] (1+\vwidth,\rparam+2*\hwidth) -- (1+\vwidth,\rparam+\hpleatwidth+2*\hwidth);
		\draw[line width = 1pt] (1+2*\vwidth,\hwidth) -- (1+2*\vwidth,\rparam+\hpleatwidth+2*\hwidth);
		\draw[line width = 1pt] (0,-\hpleatwidth/2) -- (0,\rparam+\hwidth);
		\draw[line width = 1pt] (-\vpleatwidth/2,\rparam+2*\hwidth) -- (1+\vwidth,\rparam+2*\hwidth);
		\draw[line width = 1pt] (\vwidth,-\hpleatwidth/2) -- (\vwidth,0);
		\draw[line width = 1pt] (-\vpleatwidth/2,\rparam+\hwidth) -- (0,\rparam+\hwidth);
		\begin{scope}[xscale=-1,yscale=1,shift={(\vpleatwidth-\xunitsize,0)}]{
			\draw[line width = 1pt] (\vwidth,0) -- (0,\rparam+\hwidth) -- (1+\vwidth,\rparam+2*\hwidth) -- 		(1+2*\vwidth,\hwidth) -- (\vwidth,0);
			\draw[line width = 1pt] (\vwidth,0) -- (1+\vpleatwidth+2*\vwidth,0);
			\draw[line width = 1pt] (1+2*\vwidth,\hwidth) -- (1+\vpleatwidth+2*\vwidth,\hwidth);
			\draw[line width = 1pt] (1+\vwidth,\rparam+2*\hwidth) -- (1+\vwidth,\rparam+\hpleatwidth+2*\hwidth);
			\draw[line width = 1pt] (1+2*\vwidth,\hwidth) -- (1+2*\vwidth,\rparam+\hpleatwidth+2*\hwidth);
			\draw[line width = 1pt] (0,-\hpleatwidth/2) -- (0,\rparam+\hwidth);
			\draw[line width = 1pt] (-\vpleatwidth/2,\rparam+2*\hwidth) -- (1+\vwidth,\rparam+2*\hwidth);
			\draw[line width = 1pt] (\vwidth,-\hpleatwidth/2) -- (\vwidth,0);
			\draw[line width = 1pt] (-\vpleatwidth/2,\rparam+\hwidth) -- (0,\rparam+\hwidth);
		}\end{scope}
		\begin{scope}[xscale=1,yscale=-1,shift={(0,\hpleatwidth-\yunitsize)}]{
			\draw[line width = 1pt] (\vwidth,0) -- (0,\rparam+\hwidth) -- (1+\vwidth,\rparam+2*\hwidth) -- 		(1+2*\vwidth,\hwidth) -- (\vwidth,0);
			\draw[line width = 1pt] (\vwidth,0) -- (1+\vpleatwidth+2*\vwidth,0);
			\draw[line width = 1pt] (1+2*\vwidth,\hwidth) -- (1+\vpleatwidth+2*\vwidth,\hwidth);
			\draw[line width = 1pt] (1+\vwidth,\rparam+2*\hwidth) -- (1+\vwidth,\rparam+\hpleatwidth+2*\hwidth);
			\draw[line width = 1pt] (1+2*\vwidth,\hwidth) -- (1+2*\vwidth,\rparam+\hpleatwidth+2*\hwidth);
			\draw[line width = 1pt] (0,-\hpleatwidth/2) -- (0,\rparam+\hwidth);
			\draw[line width = 1pt] (-\vpleatwidth/2,\rparam+2*\hwidth) -- (1+\vwidth,\rparam+2*\hwidth);
			\draw[line width = 1pt] (\vwidth,-\hpleatwidth/2) -- (\vwidth,0);
			\draw[line width = 1pt] (-\vpleatwidth/2,\rparam+\hwidth) -- (0,\rparam+\hwidth);
		}\end{scope}
		\begin{scope}[xscale=1,yscale=1,shift={(\xunitsize/2,\yunitsize/2)}]{
			\draw[line width = 1pt] (\vwidth,0) -- (0,\rparam+\hwidth) -- (1+\vwidth,\rparam+2*\hwidth) -- 		(1+2*\vwidth,\hwidth) -- (\vwidth,0);
			\draw[line width = 1pt] (\vwidth,0) -- (1+\vpleatwidth+2*\vwidth,0);
			\draw[line width = 1pt] (1+2*\vwidth,\hwidth) -- (1+\vpleatwidth+2*\vwidth,\hwidth);
			\draw[line width = 1pt] (1+\vwidth,\rparam+2*\hwidth) -- (1+\vwidth,\rparam+\hpleatwidth+2*\hwidth);
			\draw[line width = 1pt] (1+2*\vwidth,\hwidth) -- (1+2*\vwidth,\rparam+\hpleatwidth+2*\hwidth);
			\draw[line width = 1pt] (0,-\hpleatwidth/2) -- (0,\rparam+\hwidth);
			\draw[line width = 1pt] (-\vpleatwidth/2,\rparam+2*\hwidth) -- (1+\vwidth,\rparam+2*\hwidth);
			\draw[line width = 1pt] (\vwidth,-\hpleatwidth/2) -- (\vwidth,0);
			\draw[line width = 1pt] (-\vpleatwidth/2,\rparam+\hwidth) -- (0,\rparam+\hwidth);
		}\end{scope}
		}\end{scope}
	}
}
\end{scope}
\end{tikzpicture}
\end{center}
\end{figure*}

\end{document}